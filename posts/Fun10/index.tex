% Options for packages loaded elsewhere
\PassOptionsToPackage{unicode}{hyperref}
\PassOptionsToPackage{hyphens}{url}
\PassOptionsToPackage{dvipsnames,svgnames,x11names}{xcolor}
%
\documentclass[
  letterpaper,
  DIV=11,
  numbers=noendperiod]{scrartcl}

\usepackage{amsmath,amssymb}
\usepackage{lmodern}
\usepackage{iftex}
\ifPDFTeX
  \usepackage[T1]{fontenc}
  \usepackage[utf8]{inputenc}
  \usepackage{textcomp} % provide euro and other symbols
\else % if luatex or xetex
  \usepackage{unicode-math}
  \defaultfontfeatures{Scale=MatchLowercase}
  \defaultfontfeatures[\rmfamily]{Ligatures=TeX,Scale=1}
\fi
% Use upquote if available, for straight quotes in verbatim environments
\IfFileExists{upquote.sty}{\usepackage{upquote}}{}
\IfFileExists{microtype.sty}{% use microtype if available
  \usepackage[]{microtype}
  \UseMicrotypeSet[protrusion]{basicmath} % disable protrusion for tt fonts
}{}
\makeatletter
\@ifundefined{KOMAClassName}{% if non-KOMA class
  \IfFileExists{parskip.sty}{%
    \usepackage{parskip}
  }{% else
    \setlength{\parindent}{0pt}
    \setlength{\parskip}{6pt plus 2pt minus 1pt}}
}{% if KOMA class
  \KOMAoptions{parskip=half}}
\makeatother
\usepackage{xcolor}
\setlength{\emergencystretch}{3em} % prevent overfull lines
\setcounter{secnumdepth}{-\maxdimen} % remove section numbering
% Make \paragraph and \subparagraph free-standing
\ifx\paragraph\undefined\else
  \let\oldparagraph\paragraph
  \renewcommand{\paragraph}[1]{\oldparagraph{#1}\mbox{}}
\fi
\ifx\subparagraph\undefined\else
  \let\oldsubparagraph\subparagraph
  \renewcommand{\subparagraph}[1]{\oldsubparagraph{#1}\mbox{}}
\fi


\providecommand{\tightlist}{%
  \setlength{\itemsep}{0pt}\setlength{\parskip}{0pt}}\usepackage{longtable,booktabs,array}
\usepackage{calc} % for calculating minipage widths
% Correct order of tables after \paragraph or \subparagraph
\usepackage{etoolbox}
\makeatletter
\patchcmd\longtable{\par}{\if@noskipsec\mbox{}\fi\par}{}{}
\makeatother
% Allow footnotes in longtable head/foot
\IfFileExists{footnotehyper.sty}{\usepackage{footnotehyper}}{\usepackage{footnote}}
\makesavenoteenv{longtable}
\usepackage{graphicx}
\makeatletter
\def\maxwidth{\ifdim\Gin@nat@width>\linewidth\linewidth\else\Gin@nat@width\fi}
\def\maxheight{\ifdim\Gin@nat@height>\textheight\textheight\else\Gin@nat@height\fi}
\makeatother
% Scale images if necessary, so that they will not overflow the page
% margins by default, and it is still possible to overwrite the defaults
% using explicit options in \includegraphics[width, height, ...]{}
\setkeys{Gin}{width=\maxwidth,height=\maxheight,keepaspectratio}
% Set default figure placement to htbp
\makeatletter
\def\fps@figure{htbp}
\makeatother

\KOMAoption{captions}{tableheading}
\makeatletter
\makeatother
\makeatletter
\makeatother
\makeatletter
\@ifpackageloaded{caption}{}{\usepackage{caption}}
\AtBeginDocument{%
\ifdefined\contentsname
  \renewcommand*\contentsname{Table of contents}
\else
  \newcommand\contentsname{Table of contents}
\fi
\ifdefined\listfigurename
  \renewcommand*\listfigurename{List of Figures}
\else
  \newcommand\listfigurename{List of Figures}
\fi
\ifdefined\listtablename
  \renewcommand*\listtablename{List of Tables}
\else
  \newcommand\listtablename{List of Tables}
\fi
\ifdefined\figurename
  \renewcommand*\figurename{Figure}
\else
  \newcommand\figurename{Figure}
\fi
\ifdefined\tablename
  \renewcommand*\tablename{Table}
\else
  \newcommand\tablename{Table}
\fi
}
\@ifpackageloaded{float}{}{\usepackage{float}}
\floatstyle{ruled}
\@ifundefined{c@chapter}{\newfloat{codelisting}{h}{lop}}{\newfloat{codelisting}{h}{lop}[chapter]}
\floatname{codelisting}{Listing}
\newcommand*\listoflistings{\listof{codelisting}{List of Listings}}
\makeatother
\makeatletter
\@ifpackageloaded{caption}{}{\usepackage{caption}}
\@ifpackageloaded{subcaption}{}{\usepackage{subcaption}}
\makeatother
\makeatletter
\@ifpackageloaded{tcolorbox}{}{\usepackage[many]{tcolorbox}}
\makeatother
\makeatletter
\@ifundefined{shadecolor}{\definecolor{shadecolor}{rgb}{.97, .97, .97}}
\makeatother
\makeatletter
\makeatother
\ifLuaTeX
  \usepackage{selnolig}  % disable illegal ligatures
\fi
\IfFileExists{bookmark.sty}{\usepackage{bookmark}}{\usepackage{hyperref}}
\IfFileExists{xurl.sty}{\usepackage{xurl}}{} % add URL line breaks if available
\urlstyle{same} % disable monospaced font for URLs
\hypersetup{
  pdftitle={A Mathematical Exploration of Pizza Sizes},
  pdfauthor={Giorgio Luciano and ChatGPT4o},
  colorlinks=true,
  linkcolor={blue},
  filecolor={Maroon},
  citecolor={Blue},
  urlcolor={Blue},
  pdfcreator={LaTeX via pandoc}}

\title{A Mathematical Exploration of Pizza Sizes}
\author{Giorgio Luciano and ChatGPT4o}
\date{5/17/24}

\begin{document}
\maketitle
\ifdefined\Shaded\renewenvironment{Shaded}{\begin{tcolorbox}[borderline west={3pt}{0pt}{shadecolor}, breakable, sharp corners, interior hidden, enhanced, boxrule=0pt, frame hidden]}{\end{tcolorbox}}\fi

\hypertarget{introduction}{%
\section{Introduction}\label{introduction}}

Pizza, a beloved culinary delight, comes in various sizes. To better
understand the implications of pizza size on the amount of pizza
consumed, we establish a new standard unit called the \emph{standard
pizza radius}, denoted by the letter \(a\), which measures 6 inches.
This article examines how the area of a pizza changes with size and
demonstrates that one extra-large pizza can provide more pizza than two
standard-sized pizzas.

\hypertarget{the-area-of-a-standard-pizza}{%
\section{The Area of a Standard
Pizza}\label{the-area-of-a-standard-pizza}}

The area of a pizza, approximated as a circle, with a radius of one
standard pizza radius (\(a\)) is given by the formula:

\[
\text{Area}_{\text{standard}} = \pi a^2
\]

\hypertarget{the-area-of-an-extra-large-pizza}{%
\section{The Area of an Extra-Large
Pizza}\label{the-area-of-an-extra-large-pizza}}

For an extra-large pizza with a radius \(r = 1.5a\), the area can be
calculated as follows:

\[
\text{Area}_{\text{extra-large}} = \pi (1.5a)^2 = \pi \cdot 1.5^2 \cdot a^2 = \pi \cdot 2.25 \cdot a^2
\]

\hypertarget{comparison-two-standard-pizzas-vs.-one-extra-large-pizza}{%
\section{Comparison: Two Standard Pizzas vs.~One Extra-Large
Pizza}\label{comparison-two-standard-pizzas-vs.-one-extra-large-pizza}}

The combined area of two standard pizzas with radius \(a\) is:

\[
\text{Area}_{\text{two standard}} = 2 \cdot \pi a^2
\]

Comparing this to the area of one extra-large pizza:

\[
\pi \cdot 2.25 \cdot a^2 > 2 \cdot \pi a^2
\]

Simplifying, we find:

\[
2.25 \pi a^2 > 2 \pi a^2
\]

Thus, the area of one extra-large pizza is greater than the combined
area of two standard pizzas.

\hypertarget{minimum-radius-for-extra-large-pizza-to-always-be-larger}{%
\section{Minimum Radius for Extra-Large Pizza to Always Be
Larger}\label{minimum-radius-for-extra-large-pizza-to-always-be-larger}}

To determine the minimum radius \(r = n \cdot a\) for the extra-large
pizza to always have a greater area than two standard pizzas, we start
with the inequality:

\[
\pi (n \cdot a)^2 > 2 \cdot \pi a^2
\]

Simplifying, we get:

\[
n^2 \cdot \pi a^2 > 2 \cdot \pi a^2
\]

\[
n^2 > 2
\]

\[
n > \sqrt{2}
\]

\[
n > 1.4142
\]

Therefore, the radius of the extra-large pizza must be at least
\(\sqrt{2}\) times the radius of a standard pizza to ensure its area is
always greater than that of two standard pizzas.

\hypertarget{a-practical-example-pizza-napoletana}{%
\section{A Practical Example: Pizza
Napoletana}\label{a-practical-example-pizza-napoletana}}

In Italy, according to the \emph{Disciplinare verace pizza napoletana}
(guidelines for authentic Neapolitan pizza), the radius of a pizza
ranges from 22 to 35 cm. Let's compare the area of two pizzas with a 22
cm radius to one pizza with a 33 cm radius.

\begin{itemize}
\tightlist
\item
  \textbf{Two pizzas with 22 cm radius:}
\end{itemize}

\[
2 \cdot \pi \cdot 22^2 = 2 \cdot \pi \cdot 484 = 2 \cdot 1520.56 = 3039.52 \, \text{cm}^2
\]

\begin{itemize}
\tightlist
\item
  \textbf{One pizza with 33 cm radius:}
\end{itemize}

\[
\pi \cdot 33^2 = \pi \cdot 1089 = 3419.46 \, \text{cm}^2
\]

This calculation confirms that one pizza with a 33 cm radius has a
greater area than two pizzas with a 22 cm radius. Therefore, it is
mathematically established that consuming one extra-large pizza results
in more pizza than consuming two smaller ones.



\end{document}
